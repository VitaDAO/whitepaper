\documentclass[12pt,letterpaper]{article}

\usepackage[utf8]{inputenc}			% Sets encoding to UTF-8
\usepackage{times}					% Use Times New Roman as the default font
\usepackage{setspace}				% Free document from ugly single spacing
\usepackage{graphicx}				% So image use is possible
\usepackage{enumerate}				% Allows use of specified list numbering (letters)
\usepackage{lastpage}				% Gives the last page number as an accessible number
\usepackage{fancyhdr}				% Makes "1 of 10" type page numbering possible
\usepackage[protrusion=true,
            expansion=true,
            final]{microtype}		% Improves hbox handling, reducing errors
\usepackage{url}					% Provides clickable URLs (in the bibilography)
\usepackage[sort&compress]{natbib}	% Need this for the bibliography to work properly
\usepackage[margin=1in]{geometry}	% Set margins to standard 1 inch margins
\usepackage{xurl}
\usepackage[breaklinks,
			colorlinks = true,
            linkcolor = blue,
            urlcolor  = blue,
            citecolor = blue,
            anchorcolor = blue]
            {hyperref}				% Makes TOC Links to content
%\usepackage{showframe}				% Uncomment for troubleshooting layout issues

% use the same font for URLs - better than courier...
\urlstyle{same}

% make the text easier to read
\onehalfspacing

% don't want page numbering until the abstract
\pagenumbering{gobble}

% setting up the "1 of 10" type page numbering
\pagestyle{fancy}

% remove the "fancy" header - too cluttered and not in original
\fancyhead{} %remove contents of header
\renewcommand{\headrulewidth}{0pt} % give header no space and thus remove horizontal line

% add the "1 of 10" type page numbering
\fancyfoot[C]{\begin{flushright}\begin{footnotesize}{\thepage} of \pageref{LastPage}\end{footnotesize}\end{flushright}}

% some aliases for organizations in the bibliography who need citations to show up in the references section
\defcitealias{Ocean2021}{Ocean Marketplace}
\defcitealias{Balancer2021}{Balancer Liquidity Bootstrapping Pools}

\begin{document}

\begin{titlepage}
\centering
\includegraphics[width=\linewidth]{images/VitaDAO Opengraph.png} 
\\
\vspace{36pt}
{\begin{Large} \textbf{Whitepaper V0.1}\end{Large}}
\\
\vspace{18pt}
{\begin{large}{Authors: Tyler Golato, Paul Kohlhaas}\end{large}}
\\
\vspace{18pt}
{\begin{small}{Acknowledgments:  Jesse Hudson, Elad Verbin, James Brodie, Trent McConaghy, Robert Miller, Dimitri De Jonghe, Aubrey De Grey, Morten Scheibye-Knudsen, Theodor Walker}\end{small}}
\\
\vspace{18pt}
{\small\emph{(Draft open for public and community comment)}}
\end{titlepage}

\newpage

%start page numbering on the abstract page
\pagenumbering{arabic}

\begin{abstract}
% make the abstract section number be zero
\addtocounter{section}{0}
% add the abstract to the TOC, preserving section numbering
\addcontentsline{toc}{section}{\protect\numberline{\thesection}Abstract}

VitaDAO is a new cooperative vehicle for community-governed and decentralized drug development. Our core mission is the acceleration of R\&D in the longevity space and the extension of human life- and health-span. Today, the biopharma industry is booming with unprecedented late-stage investment inflows, particularly in the longevity space. However, critical early-stage funding is severely lacking and incentives between patients, researchers and industry are misaligned. To align incentives and vitalize early-stage funding in longevity biopharma, VitaDAO utilizes a combination of novel governance frameworks and decentralized autonomous organizations (DAOs), non-fungible tokens (NFTs), and financial engineering tools such as algorithmic automated market makers (AMMs) that run on the Ethereum blockchain.

At its core, biopharma value creation is composed of intellectual property (IP) rights, domain specific know-how and research data. Research and development has become prohibitively expensive and siloed, partially due to how IP business models incentivize monopolization of innovation through creation of thickets of patents in protected portfolios. These IP mechanisms prevent the open sharing of research data, inherently disincentivizing collaboration and transparency. They prevent the public and patients from having any real ownership in biopharma IP, even though their tax dollars fund much of the early-stage development. Outside of grants to fund basic research, early-stage funding for drug development is extremely limited. When drugs do finally make it to market, there are strong incentives for price gouging.

VitaDAO solves some of these problems. An open cooperative that anyone can join and support, VitaDAO’s goal is to acquire, support, and finance new therapeutics and research data in the longevity space, and will directly hold IP rights in novel early-stage therapeutics. Our mission is to grow a portfolio of both IP and data assets, which can be made available and monetized in novel data marketplaces such as \citetalias{Ocean2021}, promoting both open science and new business models.

Ownership and governance of VitaDAO is relegated through VITA tokens. VITA can be obtained by contributing work, or funds, or other resources such as data and IP  to VitaDAO. Ownership of VITA allows the holder to participate in the democratic governance of VitaDAO, directing its research, accessing and monetizing its data repositories, and managing its IP portfolio.

\end{abstract}

\newpage
% Make it read "Table of Contents" and center
\renewcommand{\contentsname}{\centering Table of Contents}

% Keep the TOC simple
\tableofcontents

\newpage

\section{Introduction}
Current biopharma business models carry severe limitations and R\&D inefficiencies at the cost of those who should be the core stakeholders: patients in need of medication and researchers discovering new molecules.

\subsection{Longevity Will Transform Our Approach To Medicine}
The longevity field has the power to transform our global healthcare system completely. Age-related diseases are among the most significant contributors to our global health burden and are responsible for the highest costs incurred by healthcare systems \citep{WHO2011}. 

The longevity space is experiencing a tremendous boom, in part because of its ability to address this problem. The space is evolving rapidly in terms of understanding the aging process and its capacity to develop health and life-extending therapeutics \citep{ARDD2020}. It is now a routine procedure to reverse the aging of human cells in the laboratory. Correspondingly, the economics of investing in longevity are more attractive than ever. There are 100s of new companies and funds dedicated to research and product development in the fields of senolytics, telomeres, stem cells, mitochondria, and gene editing \citep{Pfleger2021}. These are some of the core domains from which therapies that increase lifespan will emerge. However, aging is not recognized as a disease, and this has created barriers to prescribing drugs for aging itself \citep{Suresh2014}. 

The rush of investment into the field has the potential to rapidly accelerate these developments but also comes with distinct problems. The centralization of aging research by large institutions and billionaires has the potential to create the same problems and pitfalls that plague the pharmaceutical industry: lack of transparency, restricted access, and concentrated control over therapeutics that should be made widely available to anyone at an affordable price. 

We believe the future of health and life-extension should be open, collaborative, and community-owned. We see longevity as an entirely new approach to medicine, one that has the potential to prevent, as opposed to treat or intervene. This approach will completely change the way the world provides medical care and will profoundly impact society.

\subsection{Intellectual Property in Biopharma}
Fundamentally, value creation in biopharma is driven by two core assets: intellectual property (IP) and research and development (R\&D) data \citep{Chandra2011}. When companies explore a use case for a new therapeutic, they patent their discoveries to gain IP. IP enables them to secure future revenue and recoup the high costs of R\&D. As companies invest billions of dollars into generating data to bring therapeutics to market, they offset this by selling equity ownership or IP royalty rights in the form of future drug sales or milestone based payments. \citep{Wouters2020}. While IP is designed to incentivize innovation in theory, IP ownership as a business model has barely evolved in the past century, remaining largely analog with extensive legal contracts and bureaucratic complexities. IP is one of the most valuable asset classes in the world but it remains rigid, largely illiquid, and difficult to transfer.

IP owners and sellers seeking to monopolize commercial value are not incentivized to report negative research data. This creates a principal-agent dilemma and, in part, leads to a phenomenon coined the “reproducibility crisis” \citep{Sherkow2017}. Pharma tends to only share positive R\&D data which leads companies to waste enormous resources repeating each others’ failures because without complete data, clinical and preclinical studies are often irreproducible. 

Under the current system, incentives for open science, collaboration, and data sharing are severely lacking \citep{Ali-Khan2017}. Side effects or indicators of lower effectiveness of the owned compounds are downplayed in favor of profit maximization, and valuable research findings are lost for the broader scientific community and not considered during patient treatment. Monopolistic IP ownership within biopharma companies leads to centralization, disincentivizing collaboration, and promising drug candidates often fall to the wayside for organizational, political, or competitive reasons. This concentration of research efforts leads to the high R\&D cost and long time-to-market that pharma is known for today, ranging from \$800 million to \$2.6 billion on average in cost and upwards of 10 years time to market \citep{DiMasi2016}. On a macroscopic scale, this IP model leads to incentive misalignment and information asymmetry. 

Lastly, the current financing and investment landscape centralizes and restricts access and participation to early-stage biopharma from its most important stakeholders: patients and researchers. In the US, generally only accredited investors can finance early-stage ventures \citep{Rowley2018}. The market is opaque, and no singular or efficient marketplace exists to discover or access these opportunities in a transparent fashion.

More democratic, open, and transparent models are needed. Collaboration and open research in the scientific community could bring down the rising cost of drug development \citep{Bookbinder2020}.

\subsection{The Evolution of Pharma and the Valley of Death}
The process of drug development in the pharmaceutical industry is undergoing a significant shift in its industrial organization. Increasingly, smaller biotech firms leverage recent academic research in the life sciences to develop new drugs, which are then acquired by pharmaceutical giants using their access to low-cost capital to purchase expertise \citep{Huang2021} Notably, much of this academic work is funded by grants and public capital, and then privatized by pharmaceutical companies without the public receiving any upside in the research it funded \citep{Toole2004}. The number of biotech startup companies formed due to technology licensing has shifted dramatically in recent years, from 145 in 1994 to 278 in 2000 to 1,024 in 2016. The number is increasing on an exponential curve. Of the 30 top-selling drugs worldwide in 2000, only five were traceable to universities and the remaining 25 were developed by big pharma. But by 2018, more than one-half of the top 30 drugs were sourced from academia \citep{Huang2021}. Biotech universities are increasingly proving to be untapped goldmines of valuable IP, yet often these assets are not investable until a startup is spun out. 

Assets from universities that are not out-licensed or spun out into startups remain in a difficult limbo, known as The Valley of Death, where they often fail to move forward \citep{Seyhan2019}. Many of these technologies are extremely valuable to patients, and because of this system, they will ultimately never serve the populations they were intended to. These assets require investment and capital injections at an earlier stage and new open commercialization models are desperately needed to incentivize their development.

END with: this is a massive untapped resource, we need to create a creator economy for researchers and universities. They are an untapped goldmine and humanity’s biggest asset.

\section{Solution}
\textbf{We propose the creation of VitaDAO: the first fully open and community-owned, transparent intellectual property collective in longevity research.}

Members of the public can become co-owners of VitaDAO and its IP by purchasing VITA through contributing funds, valuable research data, IP assets, or services to VitaDAO. Ownership and governance of VitaDAO is controlled via VITA tokens. Ownership of VITA allows the holder to participate in the democratic governance of VitaDAO, directing its research, accessing and monetizing its data repositories, and managing its IP portfolio.

VitaDAO will acquire IP rights and R\&D data and commission research to further develop its assets. VitaDAOs core assets can consists of:
\begin{enumerate}
\item IP, including patents and licenses to therapeutics.
\item R\&D data assets generated by funding research projects.
\item Funds stored in its treasury, including unissued VITA.
\end{enumerate}

In order to acquire and finance specific research proposals, VitaDAO members may elect to issue tokens on a per-project basis, using automatic market makers (AMMs) like \citetalias{Balancer2021} (LBPs) that enable a fair and transparent participation for prospective members. 

In order to monetize and manage IP and data assets, VitaDAO members may elect to attach them to Non-Fungible Tokens (NFTs) as detailed below in Section 4.

\subsection{Stakeholders}
VitaDAO will rely on service providers (initially its advisers, Molecule, and other web3 development teams) in facilitating various administrative and developmental functions pertaining to the purchase and transfer of IP from universities, updating and maintaining the dapp, validating and formatting information selected by members, and handling other interactions that may emerge over its initial genesis and lifecycle.

\vspace{15pt}
\begin{center}
\includegraphics[width=\linewidth]{images/VitaDAO Diagram - Ecosystem-p-1600.png} 
\end{center}
\section{Core Architecture}

\subsection{Organization and Governance}
VitaDAO is an unincorporated association consisting of several core components including:
\begin{enumerate}
\item VitaDAO’s governance smart contracts, which are modeled after Moloch DAO and the LAO.
\item VitaDAO’s contribution and liquidity bootstrapping mechanism utilizing Balancer LBPs.
\item VitaDAO’s IP and data assets, which are represented as NFTs.
\end{enumerate}
VitaDAO’s smart contract architecture enables VITA tokenholders to freely engage in governance decisions pertaining to the assets and funds held by VitaDAO. VitaDAO will be managed by its members, who will use a dapp to interact with VitaDAO’s smart contracts and facilitate the purchase, funding, and management of IP and data assets.

VitaDAO members will not only govern its existing structure but also vote on how to refine its structure and governance processes. Because VitaDAO members are fully in control of the governance of VitaDAO, they not only vote but also make proposals to vote on. Proposals consist of several types: governance proposals, project proposals, funding proposals, data monetization proposals, and IP proposals, expanded on as determined by VitaDAO members.

To provide the level of industry expertise required to be able to effectively and thoroughly evaluate therapeutics projects, VitaDAO will elect working groups to review its projects, data and IP. These working groups assists in submitting proposals to VitaDAO members.

The initial goal of VitaDAO is to mimic successful execution patterns in the biotech industry, while improving on those that do not work well. Over time, we expect further governance mechanisms to emerge around VitaDAO.

In summary:
\begin{enumerate}
\item VitaDAO members are in full control of the governance of VitaDAO via votes and proposals.
\item VitaDAO will elect working groups to review its projects, data and IP. 
\item VitaDAO may engage service providers spanning patent attorneys, contract research firms or third party experts.
\item VitaDAO’s initial goal is to mimic successful execution patterns in the biotech industry.
\end{enumerate}

\subsection{"Genesis Bootstrapping and Liquidity"}
For its initial genesis, VitaDAO will deploy a Balancer LBP to enable interested participants to contribute funds to its treasury and join the collective. As VitaDAO grows, members may elect to create additional LBPs to on-board further projects.

We anticipate that the initial genesis will operate as follows:
\begin{enumerate}
\item VitaDAO will allocate a small amount of its fixed token supply into a Balancer LBP with the goal to raise sufficient funds to finance the first research project for two years. The community will decide on the full fundraising budget and structure pending simulations and crypto economic modeling using cadCAD.
\item The LBP allows the community to contribute funds and join VitaDAO for the first time. Simultaneously, the pool acts as an important first price discovery mechanism. Individuals wishing to join may contribute a multitude of assets, such as ETH, USDC or Ocean and receive VITA tokens according to the corresponding dollar value. The primary currency of the LBP is envisioned to be USDC to ensure price stability, but will ultimately be decided by members.
\item After a fixed period, the LBP resolves, forwarding its acquired liquidity to VitaDAO and kick-starting its operation. Should the LBP fail to reach a specific threshold participants may withdraw their stakes.
\end{enumerate}

\section{Holding and Managing IP and Data Assets as NFTs}

\subsection{Intellectual Property Rights}
VitaDAO will directly hold IP rights in underlying research projects and may develop a growing portfolio of assets represented as NFTs based on a novel IP NFT framework developed by Molecule, Molecule’s IP to NFT legal framework and interface allows the holder of an existing piece of IP (in the form of a license or patent) to attach it to an NFT and transfer it to a new owner.

When the IP NFT is created, the creator must sign a cryptographic message that prints their signature onto an IP licensing agreement. When the NFT is purchased, the transaction requires both buyer and seller of the NFT to sign a message that references the IP license agreement, which includes references to the transaction, the buyer and seller’s respective identities, and legal signatures modeled after the legal Signature Algorithm pioneered by \citet{OpenLaw2019}.

%ADD SECTION ON NFT fractionalisation, sharding, royalty streams. 
%https://tokeneconomy.co/token-bonding-curves-in-practice-3eb904720cb8

Initially, VitaDAO will hold IP in research projects for longevity therapeutics in the form of licenses. These licenses enable VitaDAO or its designated service providers to file for patents, monetize the IP, and contract researchers to grow the value of the underlying assets.

\subsection{Data Assets}
Besides owning IP rights related to individual projects, VitaDAO will also own data assets and research outputs from the R\&D it funds. VitaDAO will custody these assets, make them available to its members and others, and monetize them via data marketplaces such as Ocean Marketplace. VitaDAO may attach these data assets to NFTs where useful and applicable. 

VitaDAO’s introduction of its data assets into Web3 will drive fluid price discovery and liquidity for each individual data set as buyers and sellers engage around individual data sets and curate the most valuable findings. Obfuscation of data assets and enabling secure computation on them will be a highly relevant future use case that could be facilitated by Web3 protocols such as Ocean Protocol. The applications around composability and interoperability here are manifold and will be fully explored once VitaDAO is live.

Data assets could include, but are not limited to: laboratory updates and reports, processed data, quarterly updates, western blot data, qPCR data, cell culture experiments, survival analysis, sequence data, western blot gels, microscopy and cell culture. These data assets could be in the forms of raw data CSV files, images or PDFs.

The type of data assets and their monetization potential are summarized below:
\begin{enumerate}
\item \textbf{Public Data Assets:} all data in these assets are view-able by members, which can vote to list these data packages on Ocean Marketplace.
\item \textbf{IP Data Assets:} Required for filing of patents or further IP. These assets will be obfuscated using zero knowledge proofs as they could compromise filing of IP (SOURCE). These assets could be uploaded to Ocean Marketplace, as soon as IP has been filed to protect them.
\item \textbf{Progress Data Assets:} intended to keep DAO members up-to-date on ongoing research findings and serving as proof of progress. This data has a certain level of obfuscation, meaning that any information that could compromise the intellectual property itself will be hidden. Such data would not be relevant for Ocean Marketplace.
\end{enumerate}
Members wishing to engage with the IP, review the data assets directly, or engage with working groups may need to submit KYC documentation and sign non-compete and confidentiality agreements to help ensure the viability of the assets for all members.

\section{Staking and Signaling}
VITA tokens can be used to signal long-term support for individual projects. A member may stake their VITA tokens for a set time period until a project milestone is reached (e.g. 3 months/6 months/12 months). Project milestones are predetermined by projects and measurable. However, their outcome is uncertain, meaning members take additional risk by foregoing liquidity.

When a milestone is reached, staked tokens are unlocked. In return for locking tokens, VitaDAO pays out regular signaling rewards to stakers. This process sends valuable signals to the market, as it showcases the confidence of members in specific projects. Furthermore, it reduces volatility and creates a circular economy to members, which actively contribute value and review project success.

We anticipate that signaling will be implemented after the first successful project has run an initial funding cycle, giving members time to sufficiently model this mechanism and implement it in VitaDAO V2.

\section{Genesis \& First Project Lifecycle}
The following section details the initial Genesis project acquisition and models out a first potential project lifecycle. After successful completion of its Genesis Liquidity Bootstrapping and instantiation of its governance mechanism, VitaDAO members will vote to incorporate the first longevity therapeutic into its collective by purchasing an NFT attached to IP of longevity projects with the Scheibye-Knudsen Laboratory at the University of Copenhagen. Once the acquisition is complete, members can vote to allocate a first tranche of funding to the University of Copenhagen to commence research.

The project lifecycle could operate as follows:
\begin{enumerate}
\item VitaDAO proposes to purchase the Scheibye-Knudsen Laboratory IP NFT.
\item VitaDAO members vote on the proposal and define the project parameters and payment schedules with the Scheibye-Knudsen Laboratory.
\item The project commences as VitaDAO completes a first payment to initiate the preclinical studies.
\item The Scheibye-Knudsen Laboratory begins delivering first data assets consisting of progress reports, raw data and initial findings.
\item First definitive findings are in and VitaDAO members, with the support of industry advisers, decide on how to use the data and further commercialise the asset. This could result in:
\begin{enumerate}[a.]
\item Filing for further IP and patents.
\item Filing for clinical trials and further studies.
\item Monetizing the data sets through Ocean Marketplace.
\end{enumerate}
\item Simultaneously, VitaDAO may decide to launch further LBPs for additional projects at any point, which are sourced through its community, or to raise additional funding for the Scheibye-Knudsen Laboratory.
\end{enumerate}

\section{Exit Scenarios and IP Commercialisation}
If the research data surrounding an asset is sufficiently positive and clinical trials are attractive, VitaDAO can license or sell its IP or data to institutional stakeholders, such as pharmaceutical or biotech companies. Given enough liquidity and governance sophistication, VitaDAO could even attempt to bring these assets to market itself via a public listing or direct commercialization. Members of VitaDAO will decide on how to distribute income generated from successful IP monetization.

Individual IP and NFTs may be spun out into separate refractionalized markets to enable more granular participation by the public. From a technical perspective, this would entail moving the NFT and associated data sets out of custodianship of VitaDAO into a sub-DAO represented by a new ERC20 token. Members of VitaDAO may elect to spin out a project to:
\begin{enumerate}
\item Enable the project and IP to raise additional funding from a wider community.
\item Create exit scenarios for VitaDAO and further decentralize project ownership.
\item Enable more granular fundability and participation on a project level.
\end{enumerate}

\section{Open Research Questions}
The mechanics described in this paper are novel, experimental, and rely on multiple moving parts, especially the creation of an active tokenized governance community. Our goal is to tackle the below questions together as a community and through multiple working groups that contain industry experts, as they will substantially improve the system’s long-term potential.
\begin{enumerate}
\item \textbf{Governance and effective collective decision-making on drug development projects:} can a decentralized community effectively govern and attract the expertise required to successfully commercialize therapeutics? Will the token-based voting mechanism described herein succeed in generating positive outcomes?
\item \textbf{Biopharma business logic:} will the biopharma industry be open to purchasing IP and patents produced by a decentralized collective, as opposed to current business practices? Will VitaDAO be able to support a sufficient number of projects to become positive-sum at scale?
\item \textbf{Legal and patent law:} can VitaDAO successfully file for further patents on the IP assets it manages or will it have to rely on third parties? In case of patent litigation against VitaDAO, is it able to protect the validity of the IP as a normal company could?
\end{enumerate}

\section{Conclusion}
The open IP and data ownership structures presented here could help to alleviate some of the most severe challenges in the current pharmaceutical and biotech pipeline and help create a future of open and democratic ownership of longevity therapeutics. 

First, IP collectives like VitaDAO could help realign incentives with neglected stakeholder groups - patients and scientists - by creating alternative democratic research structures that distribute value and governance more evenly. Imagine research on a new breakthrough insulin treatment funded by diabetics who believed in it and stand to benefit the most from it. What would this do to access and pricing?

Second, separating R\&D data creation from IP ownership helps overcome a costly principal-agent dilemma and the current reproducibility crisis. In an openly traded curation market and structure, both negative and positive data might surface much sooner and the best therapeutics succeed. Third, transparent financialization of data  and open composability with decentralized financial markets opens up entirely new and accelerated funding models to promising scientists and laboratories across the globe. 

From an industry perspective, we believe that IP cooperatives like VitaDAO could be a win-win situation for all stakeholders: biopharma, researchers, and patients. It combines traditional industry IP holding structures with novel tokenized ownership accessible by anyone wishing to contribute to the research. In doing so, it drives valuable innovation in the biopharma industry.

% Need to clear page so reference page doesn't get messed up from above
\clearpage

% Make references show up in the TOC
\addcontentsline{toc}{section}{References} 

% fix inevitable underfull and overfull hboxes due to including urls in the reference section
% this is a suboptimal fix due to a url that uses 7 query string parameters, it beat me down
\raggedright 

% Add APA style Bibliography
\bibliography{references}
\bibliographystyle{apalike} 


\end{document}